\section{Problems}\label{sec:problems}

During the writing of this thesis, several problems arose. For some of them, we can provide an answer, others may be interesting problems for further research.

\begin{claim}
	A proper (2,3)-pole constructed from a bicritical snark is not always perfect.
\end{claim}

An counterexample from the class 1A, resulting from the Petersen graph by severing an edge and removing a vertex with a distance of 1 from the edge can be seen in \cref{fig:1A-example}. Even if we require the distance between the severed edge and the removed vertex to be more than 1 (since the results are always uncolourable or from class 1A, as proved in \cref{prop:distance-one}), we can see an uncolourable counterexample resulting from the second Blanuša snark in \cref{fig:uncolourable-example}.

\begin{claim}
	All proper (2,3)-poles resulting from the Double Star snark are perfect, when the distance between the removed vertex and the severed edge is more than one.
\end{claim}

If the distance between the removed vertex and the severed edge is more than 1, the resulting proper (2,3)-pole is perfect. Otherwise, it is in the colouring class 1A. Precisely, 1080 proper (2,3)-poles are perfect and 180 are from the colouring class 1A.

\begin{problem}
	Suppose we only consider multipoles resulting from snarks by severing an edge and removing a vertex with a distance of more than 1, since adjacent edges are trivially removable. Is a proper (2,3)-pole constructed from a snark without any removable pair of edges always perfect?
\end{problem}

There are only four such snarks from the ones we have explored: the Petersen graph, the Isaacs snarks $J_5$ and $J_7$, and the Double Star snark. All of the proper (2,3)-poles resulting from these graphs, with the distance between the removed vertex and the severed edge more than 1 are indeed perfect. However we have not proved this statement, thus it can be explored in further research.

\begin{problem}
	Construct multipoles used to extend colourings of proper (2,3)-poles, which allow the resulting proper (2,3)-poles to be contained in a nontrivial snark.
\end{problem}

As mentioned in \cref{sec:classes-to-perfect}, one of the 6-poles contains a quadrilateral, so each snark of which it is a part of is trivial.

\begin{problem}
	Construct an infinite family of snarks, that produce only colourable proper (2,3)-poles.
\end{problem}

We have found several snarks producing only colourable proper (2,3)-poles: the~Petersen graph, the Isaacs snarks $J_5$ and $J_7$ and the two Loupekine snarks of order 22. This may be helpful when exploring infinite families of snarks producing only colourable proper (2,3)-poles.

\begin{problem}
	If we construct a proper (2,3)-pole from a bicritical snark in such a way, that the distance between the severed edge and the removed vertex is more than one, and both are a part of a 5-cycle (not necessarily the same), is the result always perfect?
\end{problem}

If a counterexample is found, an additional requirement of $z(G)=5$ for the snark could be imposed.