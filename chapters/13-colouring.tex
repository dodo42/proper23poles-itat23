\section{Multipole Colouring}\label{sec:multipole-colouring}

When considering $3$-edge-colouring it is convenient to regard the colours $1$, $2$, $3$ as $(0, 1)$, $(1, 0)$, $(1,1)$, respectively. In other words, we use the set of non-zero elements of the group $\mathbb{Z}_2\times\mathbb{Z}_2$, which we will denote as $\mathbb{K}$. Using the colours from $\mathbb{K}$, it is easy to see that each $3$-edge-colouring of a cubic graphs corresponds to a nowhere zero $(\mathbb{Z}_2\times\mathbb{Z}_2)$-flow and vice versa.

% \begin{definition}
% 	Let $G$ be a graph and let $\Gamma$ be a group. A nowhere zero $\Gamma$-flow of $G$ is a pair $(D,f)$, where $D$ is an orientation assignment for each edge and $f$ is a~function $f\colon E(G)\rightarrow \Gamma$ such that the following conditions hold:
	
% 	\begin{enumerate}
% 		\item For every edge $e$ in the set $E(G)$, the function $f$ assigns a non-zero element, i.e., $f(e) \neq 0$.
% 		\item For each vertex $v \in V(G)$, the value of $\sum_{e \in v^+} f(e) - \sum_{e \in v^-} f(e)$, where $v^+$ and $v^-$ represent the sets of the edges entering and exiting vertex $v$, respectively, with orientations determined by $D$, is equal to zero.
% 	\end{enumerate}
% \end{definition}

% Since we are only exploring $3$-edge-colourings, it may be convenient to define some set of colours to be used in each colouring in our work. The set we shall use is that of non-zero elements of the group $\mathbb{Z}_2\times\mathbb{Z}_2$, which we will denote as $\mathbb{K}$.
These colourings are called \textit{Tait colourings}. These are widely used in many articles about snarks and 3-edge-colourability in general because the problem of colourability is equivalent to the~problem of finding a nowhere zero $(\mathbb{Z}_2\times\mathbb{Z}_2)$-flow. Since each non-zero element from this group is self-inverse, we need not assign an orientation to the edges.
% The~only way to achieve a zero-sum in a vertex using this set is to add three distinct values. If we were to use two identical values, we would need a third value to be zero, but since we only use non-zero elements, this is impossible. Also, the sum of the non-zero elements indeed equals zero. From now on, by default, each mentioned colouring will be a Tait colouring if not defined otherwise. Sometimes for better readability, we will use colours $1,2,3$ instead of $(0,1),(1,0),(1,1)$, respectively.

When discussing multipoles, the definition of edge-colouring is the same. The colour of an edge end is the colour of its respective edge. The \textit{colouring set} of a $k$-pole $M(e_1,\cdots, e_k)$ is the set
$$\text{Col}(M)=\{(\phi(e_1),\cdots, \phi(e_k))~|~ \phi \text{ is a Tait colouring of } M \}.$$

% As mentioned, Tait colourings are equivalent to nowhere-zero $(\mathbb{Z}_2\times\mathbb{Z}_2)$-flows. For a~better exploration of the colours in separate connectors, we shall define what is a \textit{flow through a connector}.

For a connector $S =(e_1,\cdots,e_n)$ of $M$, the  \textit{flow through} $S$ is the value $\phi_*(S)=\sum_{i=1}^{n}\phi(e_i)$. 
A connector $S$ of a multipole $M$ is called \textit{proper} if $\phi_*(S)\neq 0$ for each Tait colouring $\phi$ of $M$. A multipole is called \textit{proper} if each of its connectors is proper.

% TODO: toto potrebujeme?
In general, for any tuple of semiedges $S = (e_1, \cdots, e_n)$ and colouring $\phi$, we use the~notation $\phi(S)$ to represent the tuple $(\phi(e_1), \cdots, \phi(e_n))$.

The fact that we can regard a colouring of a multipole as a flow has a valuable consequence that will be indispensable in our work. It is commonly known as the Parity Lemma, introduced and proved by B. Descartes in 1948.

\begin{lemma}[Parity Lemma \cite{Descartes1948}]
	Let $M$ be a $k$-pole, and let $k_1$, $k_2$ and $k_3$ be the numbers of semiedges of colour $(0,1)$, $(1,0)$ and $(1,1)$, respectively. Then $k_1\equiv k_2\equiv k_3\mod 2$.
\end{lemma}

By applying the Parity Lemma, we can conclude that any cubic graph with a bridge is not colourable. Another corollary of this lemma is that the minimum number of vertices that must be removed from a snark to obtain a colourable multipole is two \cite{MorphologyOfSmall}. The same applies to severing edges. The smallest number of edges to be severed in a~snark to obtain a colourable multipole is two. If only one edge is severed, the~resulting multipole contains two semiedges, both of which must have the same colour for it to be colourable because of the Parity Lemma. That would mean the former graph resulting from the junction of these two semiedges is not a snark since the colouring of the multipole could be extended to the colouring of the graph.

A key aspect when studying the coloring properties of multipoles derived from snarks is the removability of pairs of vertices or edges.
% \begin{definition}
	Let $G$ be a snark. A pair of its distinct vertices $\{u,v\}$ is called \textit{removable} if $R(G;\{z,v\};\emptyset)$ is not colourable; otherwise, it is called \textit{unremovable}. Similarly, for edges, a pair of distinct edges $\{ab,cd\}$ is called \textit{removable} if $R(G;\emptyset;\{ab,cd\})$ is colourable. Otherwise, it is called \textit{unremovable}.
% \end{definition}

If we sever two adjacent edges, it is equivalent to the removal of a single vertex with regard to colourability. Therefore these pairs of edges are trivially removable because at least two vertices are needed to be removed from a snark to obtain a colourable multipole.
