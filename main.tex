%% The first command in your LaTeX source must be the \documentclass command.
%%
%% Options:
%% twocolumn : Two column layout.
%% hf: enable header and footer.
\documentclass[
twocolumn,
% hf,
]{ceurart}

%%
%% One can fix some overfulls
\sloppy

%%
%% Minted listings support 
%% Need pygment <http://pygments.org/> <http://pypi.python.org/pypi/Pygments>
\usepackage{listings}
%% auto break lines
\lstset{breaklines=true}

%%
%% end of the preamble, start of the body of the document source.
\input definitions.tex
\begin{document}

%%
%% Rights management information.
%% CC-BY is default license.
\copyrightyear{2022}
\copyrightclause{Copyright for this paper by its authors.
  Use permitted under Creative Commons License Attribution 4.0
  International (CC BY 4.0).}

%%
%% This command is for the conference information
\conference{ITAT’23: Information technologies -- Applications and Theory, September 22–26, 2023, Tatranské Matliare, Slovakia}

%%
%% The "title" command
\title{Analysis of colouring properties of proper (2, 3)-poles}


%%
%% The "author" command and its associated commands are used to define
%% the authors and their affiliations.
\author[1]{Erik Řehulka}[%
email=rehulka3@uniba.sk,
]
\address[1]{Comenius University, Mlynská dolina, 842 48 Bratislava, Slovakia}

\author[1]{Jozef Rajník}[%
email=jozef.rajnik@fmph.uniba.sk,
]

%%
%% The abstract is a short summary of the work to be presented in the
%% article.
\begin{abstract}
  Snarks, that is $2$-connected cubic graphs admitting no $3$-edge-colouring, provide a promising family of cubic graph with respect to many widely-open conjectures. They are often constructed by joining several building blocks which are cubic ``graphs'' with dangling edges allowed, formally called multipoles. In our work, we explore the colouring properties of proper (2,3)-poles, a specific type of multipole with five dangling edges. They result from snarks by removing a vertex and severing an edge, not incident with the removed vertex. To conduct our analysis, we explore all proper (2,3)-poles resulting from nontrivial snarks with at most 28 vertices. This encompasses a~total of 3,247 snarks and 3,476,400 proper (2,3)-poles. In our research, we provide various structures that can be utilized to expand the colourability of proper (2,3)-poles. In the core of our work, we provide theorems regarding the colouring properties of proper (2,3)-poles, specifically necessary and sufficient conditions for these properties. Additionally, we present the data and observations from the analysis.
\end{abstract}

%%
%% Keywords. The author(s) should pick words that accurately describe
%% the work being presented. Separate the keywords with commas.
\begin{keywords}
	snark \sep 
	multipole \sep 
	edge-colouring \sep 
	Tait colouring \sep 
	colouring set
\end{keywords}

%%
%% This command processes the author and affiliation and title
%% information and builds the first part of the formatted document.
\maketitle

\input chapters/00-introduction.tex
\input chapters/10-multipolesAndSnarks.tex 
\input chapters/20-proper23poles.tex
\input chapters/40-ourWork.tex
%\input chapters/90-conclusion.tex

\bibliography{references}

\end{document}

%%
%% End of file
